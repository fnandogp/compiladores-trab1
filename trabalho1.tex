\documentclass[a4paper,10pt]{article}
\usepackage{graphicx,url}
\usepackage[brazil]{babel}
%\usepackage[latin1]{inputenc}
\usepackage[utf8]{inputenc}

\title{Compiladores - Trabalho 1}
\author{Fernando Guimarães Pinheiro}

\begin{document}

\maketitle



\section{Objetivo}
	Criação de um analizador léxico para a linguagem G-Portugol.

\section{Requisitos}

	\subsection{Arquivo de entrada}
		O arquivo de entrada deverá ser um programa em G-Portgual.

		O analisador deve reconhecer os seguintes símbolos:
		\begin{itemize}
		\item Palavra Reservada
		\item Identificador
		\item Inteiro
		\item Real
		\item Caractere
		\item Literal
		\item Lógico
		\item Operador Aritmético
		\item Operador Relacional
		\item Operador Lógico
		\item Símbolo Especial
		\item Atribuição
		\end{itemize}

		E deverão ser desconsiderados:
		\begin{itemize}
		 \item Espaços
		 \item Tabs
		 \item Enters
 		 \item Comentarios de linha e de bloco
		\end{itemize}


	\subsection{Saída}
		Para cada símbolo reconhecido da linguagem deverá ser gerada uma saída no terminal.
		Comentários não apresentam saida e símbolos não reconhecidos também são possuem saida.

	\subsection{Exemplo}
		Código de entrada:

		\begin{verbatim}
		// meu primeiro programa em g-portugol
		algoritmo primeiro;

		variáveis
			resposta: literal;
		fim-variáveis

		início

			imprima("Este e seu primeiro programa em g-portugol? (s/n)");
			resposta := leia();

			se resposta = "s" então
				imprima("Sim, este e seu primeiro programa em g-portugol!");
			senão
				imprima("Não, ja fiz outros programas em g-portugol.");
			fim-se
		fim
		\end{verbatim}

		Código de saída:

		\begin{verbatim}
		algoritmo -> PALAVRA RESERVADA
		primeiro -> IDENTIFICADOR
		; -> SIMBOLO ESPECIAL
		variáveis -> PALAVRA RESERVADA
		resposta -> IDENTIFICADOR
		: -> SIMBOLO ESPECIAL
		literal -> PALAVRA RESERVADA
		; -> SIMBOLO ESPECIAL
		fim-variáveis -> PALAVRA RESERVADA
		início -> PALAVRA RESERVADA
		imprima -> PALAVRA RESERVADA
		( -> SIMBOLO ESPECIAL
		"Este e seu primeiro programa em g-portugol? (s/n)" -> LEXICO
		) -> SIMBOLO ESPECIAL
		; -> SIMBOLO ESPECIAL
		resposta -> IDENTIFICADOR
		: -> SIMBOLO ESPECIAL
		= -> OPERADOR RELACIONAL
		leia -> PALAVRA RESERVADA
		( -> SIMBOLO ESPECIAL
		) -> SIMBOLO ESPECIAL
		; -> SIMBOLO ESPECIAL
		se -> PALAVRA RESERVADA
		resposta -> IDENTIFICADOR
		= -> OPERADOR RELACIONAL
		"s" -> LEXICO
		então -> PALAVRA RESERVADA
		imprima -> PALAVRA RESERVADA
		( -> SIMBOLO ESPECIAL
		"Sim, este e seu primeiro programa em g-portugol!" -> LEXICO
		) -> SIMBOLO ESPECIAL
		; -> SIMBOLO ESPECIAL
		senão -> PALAVRA RESERVADA
		imprima -> PALAVRA RESERVADA
		( -> SIMBOLO ESPECIAL
		"Não, ja fiz outros programas em g-portugol." -> LEXICO
		) -> SIMBOLO ESPECIAL
		; -> SIMBOLO ESPECIAL
		fim-se -> PALAVRA RESERVADA
		fim -> PALAVRA RESERVADA

		\end{verbatim}

		 


\section{Desenvolvimento}
	\subsection{Comentários}
		Foi utilizado como base o manual onde a conversão léxica era apresentada da seguinte maneira:
		\begin{verbatim}
			SL_COMMENT : "//" [^LF]* ('\n')?
			ML_COMMENT : "/*" ( ~('*') | '*' ~'/' )* "*/"
		\end{verbatim}

	\subsection{Palavras reservadas}
		Foram usadas as palavras reservadas contidas no manual da linguagem
		passado como referência.
		\begin{verbatim}
		algoritmo|variáveis|fim-variáveis|início|
		fim|inteiro|inteiros|caractere|caracteres|
		real|reais|literal|lógico|lógicos|matriz|
		se|então|senão|fim-se|para|de|até|faça|
		fim-para|passo|enquanto|fim-enquanto|
		imprima|leia|função|retorne
		\end{verbatim}

	\subsection{Lógico}
		\begin{verbatim}
		verdadeiro|falso
		\end{verbatim}

	\subsection{Inteiro}
		\begin{verbatim}
		("+"|"-")?[0-9]+ |
		0(x|X)[0-9a-fA-f]+ |
		0(c|C)[0-7]+ |
		0(b|B)[0-1]+
		\end{verbatim}

	\subsection{Real}
		\begin{verbatim}
		("+"|-)?[0-9]"."[0-9]+
		\end{verbatim}

	\subsection{Caractere}
		\begin{verbatim}
		\'([^\\]|\\.)?\'
		\end{verbatim}

	\subsection{Literal}
		\begin{verbatim}
		\"([^\"\\\n\r]|\\.)*\"
		\end{verbatim}

	\subsection{Operador Aritmético}
		%(\+|\-|\/|\*|\%|\+\+­­)
		\begin{verbatim}
		\end{verbatim}

	\subsection{Operador Relacional}
		\begin{verbatim}
		(\>|\>\=|\<|\<\=|\=|\<\>)
		\end{verbatim}

	\subsection{Operador Lógico}
		\begin{verbatim}
		&&|"||"|!|ou|e
		\end{verbatim}

	\subsection{Símbolo especial}
		\begin{verbatim}
		(\"|\(|\)|\.|\,|\;|\:|\{|\}|\#|\'|\\)
		\end{verbatim}

	\subsection{Atribuição}
		\begin{verbatim}
		:=
		\end{verbatim}

	\subsection{Espaços, Tabs e Enters}
		\begin{verbatim}
		[ \t\n]
		\end{verbatim}

	\subsection{Desconhecido}
		\begin{verbatim}
		.
		\end{verbatim}

\end{document}
